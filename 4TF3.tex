\newpage
\section{Task Force 3}
\label{chapter 4}


\subsection{Development for planetariums - DFP}
\textbf{Objective:} Develop special shows for planetariums in the region.
\\
\textbf{Leading ROAD Institution:} Parque EXPLORA (Colombia).
\\
\textbf{Contact:} Carlos Molina.
\\
\textbf{Implementation:} We want to develop special shows to be projected at planetariums and science museums from the region, highlighting ancestral traditions from indigenous tribes and founding communities from South America.
These shows would be shared between the Local and Regional Networks of Planetariums, such as “Asociación de Planetarios
del Cono Sur”, and the growing “Red de Planetarios de Colombia”.

\subsubsection{Results}

We only succeded to perform initial contacts.
We did not manage to develop the special shows.


Representatives of Parque Explora (Medellín, Colombia) got in touch with chilean colleages to develop ideas to create shows for planetariums in the Andean region. There was a plan to implement a pilot plan in 2017-2018 for a digital planetarium to be located in San Pedro de Atacama.
Unfortunately this did not came to fruition.
The National Polytechnic School in Quito had a initial approach with Planetarium in Quito. 

\subsection{Communicating astronomy with the public}
\textbf{Objective:} Organize periodic meetings to gather all the collaborators in the Task Force, "Communicating Astronomy with the Public”.
\\
\textbf{Leading ROAD Institution:} SOCHIAS ROAD office (Chile, Oficina Nacional de Coordinación, ONC).
\\
\textbf{Contact:} Farid Char.
\textbf{Implementation:} We want to hold regular meetings that can serve as an exchange platform for people involved in communicating astronomy to the public; this includes science museums, planetariums, mass media and associated industries. We also want to use this opportunity to showcase efforts in the other two task forces. This meeting will be
held parallel to large professional meetings. We envision this event as a ”Communicating Astronomy with the Public” meeting for the region. We expect to hold the first one during the next Latin-American Regional IAU Meeting in 2016.


\subsubsection{Results}
We implemented a one-afternoon mini-worskshop during the Latin-American Regional IAU Meeting in 2016.
Parque Explore hosted the Communicating Astronomy with the Public meeting in 2016. Colleagues from the Andean ROAD attended,
but we did not manage to stablish a similar regular meeting for the Andean Region. 

\newpage
\section{Task Force 3}
\label{chapter 4}

Here was exposed the tasks acquired by the TF3, related with Astronomy for the Public

\subsection{Development for planetariums - DFP}
\textbf{Objective:} Develop special shows for planetariums in the region.
\\
\textbf{Leading ROAD Institution:} Parque EXPLORA (Colombia).
\\
\textbf{Contact:} Carlos Molina.
\\
\textbf{Implementation:} We want to develop special shows to be projected at planetariums and science museums from the region, highlighting ancestral traditions from indigenous tribes and founding communities from South America. These shows would be shared between the Local and Regional Networks of Planetariums, such as “Asociación de Planetarios
del Cono Sur”, and the growing “Red de Planetarios de Colombia”.

\subsubsection{Achievements}
The National Polytechnic School had an approach with IGM Planetarium, the team needs to be repowering.
\\
From Chile was made a collaboration with representatives of Parque Explora (Medellín, Colombia) in the development of ideas to create shows for planetariums in the Andean region, using resources to adapt presentations to different geographical realities.
\\
Also, was planed to implement a pilot plan during digital planetarium development to be located in San Pedro de Atacama, between the end of 2017, the beginning of 2018.

\subsection{Communicating astronomy with the public - CAWP}
\textbf{Objective:} Organize periodic meetings to gather all the collaborators in the Task Force, "Communicating Astronomy with the Public”.
\\
\textbf{Leading ROAD Institution:} SOCHIAS ROAD office (Chile, Oficina Nacional de Coordinación, ONC).
\\
\textbf{Contact:} Farid Char.
\textbf{Implementation:} We want to hold regular meetings that can serve as an exchange platform for people involved in communicating astronomy to the public; this includes science museums, planetariums, mass media and associated industries. We also want to use this opportunity to showcase efforts in the other two task forces. This meeting will be
held parallel to large professional meetings. We envision this event as a ”Communicating Astronomy with the Public” 2 meeting for the region. We expect to hold the first one during the next Latin-American Regional IAU Meeting in 2016.

%---- the scope of this chapter}}

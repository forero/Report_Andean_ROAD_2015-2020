% License:
% CC BY-NC-SA 3.0 (http://creativecommons.org/licenses/by-nc-sa/3.0/)
%
%%%%%%%%%%%%%%%%%%%%%%%%%%%%%%%%%%%%%%%%%

%----------------------------------------------------------------------------------------
%	PACKAGES AND OTHER DOCUMENT CONFIGURATIONS
%----------------------------------------------------------------------------------------

\documentclass[paper=a4, fontsize=11pt]{scrartcl} % A4 paper and 11pt font size

\usepackage[T1]{fontenc} % Use 8-bit encoding that has 256 glyphs
\usepackage{fourier} % Use the Adobe Utopia font for the document - comment this line to return to the LaTeX default
\usepackage[english]{babel} % English language/hyphenation
\usepackage{amsmath,amsfonts,amsthm} % Math packages
\usepackage{lipsum} % Used for inserting dummy 'Lorem ipsum' text into the template

\usepackage{caption}
\usepackage{subcaption}
\usepackage{graphicx}

\usepackage{float}

\usepackage{blindtext} %for enumarations

\usepackage[]{hyperref}  %link collor

%talbe layout to the right
%\usepackage[labelfont=bf]{caption}
%\captionsetup[table]{labelsep=space,justification=raggedright,singlelinecheck=off}
%\captionsetup[figure]{labelsep=quad}

\usepackage{sectsty} % Allows customizing section commands
\allsectionsfont{\centering \normalfont\scshape} % Make all sections centered, the default font and small caps

\usepackage{fancyhdr} % Custom headers and footers

\usepackage{ragged2e}
\usepackage{hyphenat}
\usepackage[utf8]{inputenc}
\usepackage{lscape}

\pagestyle{fancyplain} % Makes all pages in the document conform to the custom headers and footers
\fancyhead{} % No page header - if you want one, create it in the same way as the footers below
\fancyfoot[L]{} % Empty left footer
\fancyfoot[C]{} % Empty center footer
\fancyfoot[R]{\thepage} % Page numbering for right footer
\renewcommand{\headrulewidth}{0pt} % Remove header underlines
\renewcommand{\footrulewidth}{0pt} % Remove footer underlines
\setlength{\headheight}{13.6pt} % Customize the height of the header

\numberwithin{equation}{section} % Number equations within sections (i.e. 1.1, 1.2, 2.1, 2.2 instead of 1, 2, 3, 4)
\numberwithin{figure}{section} % Number figures within sections (i.e. 1.1, 1.2, 2.1, 2.2 instead of 1, 2, 3, 4)
\numberwithin{table}{section} % Number tables within sections (i.e. 1.1, 1.2, 2.1, 2.2 instead of 1, 2, 3, 4)

%\setlength\parindent{0pt} % Removes all indentation from paragraphs - comment this line for an assignment with lots of text


\setlength\parskip{4pt}

%----------------------------------------------------------------------------------------
%	TITLE SECTION
%----------------------------------------------------------------------------------------

\newcommand{\horrule}[1]{\rule{\linewidth}{#1}} % Create horizontal rule command with 1 argument of height

\title{	
\normalfont \normalsize 
\textsc{} \\ [25pt] % Your university, school and/or department name(s)
\horrule{0.5pt} \\[0.4cm] % Thin top horizontal rule
\huge  Report - 4 Years of the Andean ROAD\\ % The assignment title
\horrule{2pt} \\[0.5cm] % Thick bottom horizontal rule
}

\author{Forero Romero, Jaime Ernesto} % Your name

\date{\normalsize\today} % Today's date or a custom date

\begin{document}
%\nocite{*}
\maketitle % Print the title

\newpage
\begin{abstract}

{\noindent{\Large Abstract} }\\

The main goal of the Andean Regional Office of Astronomy for Development (ROAD) is serve to develop strategies to strengthen the professional research, education and popularization of astronomy in the Andean region. It will strengthen ongoing collaboration efforts, create channels of communication and develop new strategies to exchange knowledge and human resources in the region.
\\

The tasks to achieve the proposed goal were classified in 3 task forces; TF1: Universities and Research , TF2: Astronomy for Children and Schools and TF3: Astronomy for the Public . Different institutions from the countries that composed the Andean Region (Colombia, Venezuela, Ecuador, Perú, Bolivia and Chile) got engaged with these TFs.
\\

This document collects the progress, results or difficulties of the Andean ROAD since 2014. 

\end{abstract}

\newpage
\tableofcontents

%----------------------------------------------------------------------------------------
%	Section 1
%----------------------------------------------------------------------------------------

\newpage
\section{Introduction}

In the proposal to start the Andean Regional Office of Astronomy for
Development (ROAD), submitted on November 2, 2014, we wrote the following:

\begin{quote}
The countries in the Andean Region (Bolivia, Colombia, Chile,
 Ecuador,Peru and Venezuela) represent a common language block in
 South America. 
 Historically they have also made part of common trading blogs (Andean
 Community, Comunidad Andina (CAN) in Spanish). 
They share similar social conditions and goals for scientific and
 human developmen development. 


The astronomical development in each of these countries can be
efficiently achieved through regional cooperation. 
With the leadership of Chile in aspects of professional astronomy and
Colombia for public outreach, it is possible to develop strategies to
strengthen the professional research, education and popularization of
astronomy in the Andean region. 

This development effort requires the commitment and shared effort from
different institutions. 
An Andean Regional Office of Astronomy for Development (ROAD) can
serve this goal. It will strengthen ongoing collaboration efforts,
create channels of communication and develop new strategies to
exchange knowledge and human resources in the region.  


The participating institutions have been involved in different
projects in each one of the ROAD core areas: research, schools and
children and public. However, the participation and degree of
commitment has not been homogeneous across the region, or even inside
the same country. 


Therefore our main goal is to guarantee and strengthen effective
methods of communication between the representatives and coordinators
of global, regional and local projects implemented in the Andean
countries, specially looking to orient and advise new working groups 
in other cities and regions. This lead us to define the following
objectives

\begin{itemize}
\item Foster the goals of the International Astronomical Union (IAU)
Strate- gic Plan in the Andean region. 

\item Serve as partner to the Office of Astronomy for Development (OAD),
the IAU and other international organizations to plan and implement
relevant projects in the Andean region. 
\item Create public forums where the project management of all
development activities can be communicated and evaluated. 
\item Initiate and coordinate fund-raising activities for regional
development activities. 
\item Create new institutional alliances among countries in the region
to exchange knowledge and human resources. 

\end{itemize}

\end{quote} 

To reach those objectives we suggested a series of specific projects and tasks. 
In this report we evaluate to what extent those task were succesfuly
completed.

The following score card grades each task from 0 (unfulfilled) to 10
(completely fullfilled). We give a grade of $2$ to tasks that were
started but did not continue. We give a grade of $6$ to tasks that
advanced and gave positive results only for a fraction of the
institutions participating in the Andean ROAD.

\section{Score Card}

\begin{table}[!h]
\begin{center}
\begin{tabular}{lp{9cm}r}
Task Force & Project & Score (0-10)\\\hline
1 & Andean School On Astronommy Astrophysics & 6\\
1 & Andean Graduate Program & 0 \\
1 & Exploration Working Group in Astroparticle Physics & 6\\
1 & Exploration Working Group in Radioastronomy & 6\\
1 & Massive Open Online Courses & 0 \\ 
1-2-3 & Communitacion Network & 10\\
2  & Periodic Coordination Sessions & 2\\
2 & Virtual Training Sessions & 2\\
2 & Teaching Material for Visually Impaired Students & 6\\
2 & Annual TF2 Meeting & 2\\
3 & Development for planetariums & 0\\
3 & Regional Communicating Astronomy with the Public & 2\\\hline
 & \textbf{Average} & 3.33\\ 
\end{tabular}
\end{center}

In the following section we present more details about each project.
\end{table}




%----------------------------------------------------------------------------------------
%	Section 2
%----------------------------------------------------------------------------------------

\section{Task Force 1}
\label{chapter2}
\textit{}
%---- the scope of this chapter}}


\subsection{Andean School on Astronomy and Astrophysics}
\textbf{Objective} Organize a school for advanced undergraduate
students and graduate students.  
\\
\textbf{Lead:} Changes every year. 
\\
\textbf{Planned Implementation:} 
Every year we will hold a school aimed at advanced undergraduate
students and graduate students.  
The main subjects of the school have to be broad enough allowing a
large student participation.  
This venue will also serve as a upstanding scenario to invite tutors
from abroad and strengthen new institutional collaborations with the
Andean ROAD.  
The first school will take place in 2014 hosted in Ecuador close to
the dates for the Colombian Congress of Astronomy and Astrophysics. 

\subsubsection{Results}

We managed to organize three schools in Quito (2014), Bogot\'a (2015)
and Lima (2018).  

The first school in Quito (2014) was lead by Prof. Nicol\'as V\'asquez
at the Escuela Polit\'ecnica Nacional.  
The school had [XX] students from [COUNTRIES] and [XX] instructors
from [COUNTRIES].  
The dates of the school were [XXX] and the main teaching subjects were [XXX].

The second school took place in Bogota (June 1 through June 26, 2015). 
It was lead by Prof. Jaime E. Forero-Romero at Universidad de los
Andes (\url{http://forero.github.io/AndeanCosmologySchool/}). 
The school had 18  students from Bolivia, Colombia, Chile, Ecuador,
Peru and Venezuela.
It had four instructors working in Germany (Potsdam Astrophysical Institute), 
Israel (Hebrew University) and the United States (Caltech and Berkeley Lab)
The main subject was cosmology in four aspects: theory, simulations,
experiment and intrumentation. 

The third school was held in Lima (November 12-16, 2018) 
It took place within the third workshop \emph{Astronom\'ia en los Andes}. 
It was lead by Prof. Maria Isela Zevallos from the Universidad
Nacional de Ingenier\'ia. The schoold had [XX] students from
[COUNTRIES] and [XX] instructors   
The school had the following courses

\begin{enumerate}
    \item  Milky Way Structure and Evolution. Dr. Dante Minniti Universidad Andrés Bello (UNAB, Chile).
    \item "Astronomical Techniques in Planetary Sciences". Dr. Gonzalo Tancredi (University of Uruguay, Uruguay).
    \item "Astronomy in the School". Dr. Paulo Sergio Bretones (UFSC, Brazil).
\end{enumerate}

For this school the biggest challenge was to get participation from
researchers, amateurs and students from all the Andean ROAD
countries. No participants came from Bolivia, nor from Chile.  
The most numerous participation was that of university students
followed by that of the fans and, finally, the professors. 
The scientific community in Astronomy in Peru is small, but most of
them attended.   



The biggest challenge for all schools was to secure stable funding.
 The first school relied entirely on local funding.
The second school the central OAD office sponsored the
accommodation for the 15 students from Ecuador, Venezuela, 
Bolivia and Chile, which represented close to $20\%$ of the total expenses.
For the third school the Andean ROAD sponsored the accommodation for
10 students from  Ecuador, Colombia, Venezuela, Argentina and Peru,
which represented close to $10\%$ of the total expenses. 

Putting together every school required a large effort from the Local
Organizing Committee to secure the funding from local organizations.
This instability made it difficult for us to continue with the desired
annual periodicity. 

 
%Since the website of the Third Workshop, 
%\url{http://vri.uni.edu.pe/index.php/3waa/} is not fully active.
%\newline    
%\\
%\textbf{Scientific Committee.}
%\begin{itemize}
%    \item Dr. Iván Ramírez (TCC - USA)
%    \item Dr. Wiliam Hipólito (UFES - Brasil)
%    \item Dr. Jorge Gonzáles (UNMSM - Perú)
%    \item Dr. Germán Chaparro (ECCI- Colombia)
%    \item Dra. Angela Patricia Pérez (Planetario de Medellín – Colombia)
%    \item Dr. Luis A. Nuñez (Universidad Industrial de Santander - Colombia)\\
%\end{itemize}
%\newline
%    \textbf{Local  Organized Committee.}
%    \begin{itemize}
%        \item Maria Isela Zevallos (UNI)
%        \item Julio Tello (UNI)
%        \item Vanesa Navarrete (CONIDA)
%        \item Nobar Baella (IGP)
%        \item Teófilo Vargas (UNMSM)
%        \item Angel Carranza (UNT)
%    \end{itemize}        
%\newline
%\bigskip

%On behalf of CIDA in Mérida, Venezuela, it was agreed to play a support and support role within TF1, for the tasks of the ASAA. 
%What was then offered to Andean ROAD was his experience of several years in the realization of astronomy schools within Venezuela, some of these were attended by Colombian students, and also offered the possibility of teaching some online postgraduate courses, because they had the human capital and the technical possibilities (internet) of doing so. 
%In those first meetings of the Node held in Bogotá, it was seen that the conditions of professional astronomy in Venezuela were deteriorating and several years later, that prediction has been coming true. 
%Currently, there are only four professional astronomers left in practice, and although the community in Venezuela was never very large, it was up to 4 times the current number. 
%The forecasts for next year are that one of the 4 professionals retire and the remaining 3 migrate. 
%Currently, Venezuela as a country, is not in a position to offer something concrete, however, some Venezuelan astronomers, individually, in other countries of the region, will be able to continue supporting the tasks of the TF1 of the Andean ROAD, from their respective countries of residence.

%The National Polytechnic School of Ecuador participated in the organization of the First Andean School of Astronomy and Astrophysics. (Quito, December 2014, Organizer: Nicolás Vásquez) and the Second Andean School of Astronomy and Astrophysics.

%#######################################################

\subsection{Andean Graduate Program}
\textbf{Objective} Determine the feasibility of creating and funding
an Andean graduate program. 
\\
\textbf{Leading ROAD Institutions:} Observatorio Astronómico Nacional
(Colombia); SOCHIAS ROAD Office (Chile, Oficina Nacional de
Coordinación, ONC); Universidad de San Francisco de Quito (Ecuador). 
\\
\textbf{Lead:} Giovanni Pinzón, PhD; Eduardo Unda-Sanzana, PhD;
Dennis Cazar Ramírez, PhD.   
\\
\textbf{Implementation:} Explore the possibility to coordinate the use
of educational resources of different institutions in the region. The
model for this project is the AstroMundus program in the European
Union where a consortium of 5 Universities in 4 Countries offer a
Master Program. This research will be done mainly through virtual
meetings.  


\subsubsection{Results}
The project  did not make any significant progress towards reaching its
objective.  


%#######################################################

\subsection{Exploration Working Group in Astroparticle Physics}
\textbf{Objective} Establish a cooperation network of institutions
interested in Astroparticle Physics. 
\\
\textbf{Leading ROAD Institution:} Universidad Industrial de Santander (Colombia); Universidad San Francisco de Quito (Ecuador). 
\\
\textbf{Lead:} Luis Nuñez, PhD.; Dennis Cazar Ramírez, PhD.
\\
\textbf{Implementation:} The leading institutions will organize
periodic meetings to gather all the groups in the Andean Region
interested in developing its research capabilities in astroparticle
physics. This will be done through workshops and the installations of
small research stations for the Latin American Giant Observatory
(LAGO) project, which has already been kick-started in Venezuela,
Colombia, Ecuador, Peru and Bolivia. 

\subsubsection{Results}
At the Escuela Politécnica Nacional of Quito Ecuador it was created
the \href{http://201.159.223.36/poli/index.php}{astroparticles
  laboratory}. Also, the institution continue working with the
\href{http://lagoproject.net/}{Colaboración Lago}, this collaboration
is presided by Ivan Sidelnick. %\email{ivan.sidelnik@gmail.com}. 

In Ecuador, in the National Polytechnic School was created the
Ecuatorian Network of Cosmic Rays, Astroparticles and Space Weather
with the particpitacion of the National Polytechnic School, San
Francisco University of Quito and the Chimborazo Polytechnic
School. \href{Astroparticle Laboratory of the  National Polytechnic
  School}{http://201.159.223.36/poli/index.php} 
%Another teacher of this network is working at Yachat Tech.

Also, it was organized in Quito the VII School of Cosmic Rays and
Astroparticles by Oscar Saavedra, on August 2017. 
\\
In the 2020 will be the VIII School of Cosmic Rays and Astroparticles
in Bolivia, the organizer is Martín Alfonso Subieta
%\email{martin.alfonso.subieta.vasquez@cern.ch}. 


%#######################################################


\subsection{Exploration Working Group in Radioastronomy}
\textbf{Objective} Establish a cooperation network of institutions
interested in the development of Radioastronomy. 
\\
\textbf{Leading ROAD Institution:} Escuela Colombiana de Carreas
Industriales (Colombia). 
\\
\textbf{Lead:} Germán Chaparro Molano, PhD. 
\\
\textbf{Implementation:} The Leading Institution will organize
periodic meetings to gather all the groups in the Andean Region
interested in the development of Radioastronomy capabilities. The
first meeting is planned for 2015 The Leading Institution will also
organize a Training School to build up new capabilities and strengthen
the ties between different groups and countries. The first school is
planned for 2016. 

\subsubsection{Results}
In radioastronomy was consolidated a work network between Universities of Chile and Colombia.
These universities are: 
\begin{enumerate}
    \item Industrial University of Santander (UIS) with the contact of
      Professor Julián Rodríguez 
    \item ECCI University, contact Professor Germán Chaparro and Oscar Restrepo.
    \item University of Chile, Contact Professor Patricio Mena and
      Oscar Restrepo. 
    \item Catholic University of La Santisima Concepción, Professor
      Ricardo Bustos 
\end{enumerate}

With the previous group of researchers, it was possible established a
work network and began on a joint project that was presented to the
National Spectrum Agency (ANE), this project was approved and is
already in execution. As part of the participation from Chile, the
participation of Professor Ricardo Bustos in the IX International
Congress "Spectrum for sustainable development" organized by the ANE
in September 2019 was achieved. In this participation, the network
created and the work was commented on Future to develop in Colombia in
the field of Radio astronomy. 

Besides the participation of Professor Ricardo in the Congress, there
was also participation of Professor Julián Rodríguez (UIS) who spoke
among other things about political needs as a Country for the
regulation of the use of the radio spectrum in the field of Radio
Astronomy. 



%#######################################################





\subsection{Massive Open Online Courses - MOOC}
\textbf{Objective} Create a Massive Open Online Course at the
introductory undergraduate level with sections for the general
public. \\
\textbf{Leading ROAD Institution:} Universidad Industrial de Santander
(Colombia). \\
\textbf{Lead: }Luis Nuñez, PhD  \\
\textbf{Implementation:} We will create a Massive Online Open Course
in Astronomy and Astrophysics, that could be used either as a general
course for a wide audience or as an introductory course for
undergraduate students of Physics programs. Planned for 14 weeks, the
course will contain four major blocks: Astronomy and celestial bodies,
Instrumentation in Astronomy, Current Trends in Astronomy and
Astrophysics, and Data Bases and Data analysis in Astronomy. The first
three blocks will be divided in two modules corresponding to two
different levels: A and B. General audience could follow the level A
for the three modules and physics students should follow whole content
of the course. 
Each module, will have a 15m video, recommended reading and several
assignments. 
\\ 

%#######################################################

\subsubsection{Results}

The project  did not make any significant progress towards reaching its
objective.  



\subsection{Communication Network}
\textbf{Objective} Gather contact information of all TF1 colleagues in the ROAD.
\\
\textbf{Leading ROAD Institution:} Universidad de los Andes (Colombia)
\\
\textbf{Lead:} Jaime E. Forero-Romero, PhD
\\
\textbf{Implementation:} Open a mailing list associated to TF1 activities.
\\
This network will serve as a platform to exchange information
concerning joint projects, fellowships, scholarships and open
positions. There will also be a central web-page for the Andean ROAD
showing the current ongoing projects and advances. 

\subsubsection{Results}
We created three different mailing lists, one for each Task Force. The
mailing lists continue to be active  and are the main comnunication
channel across the collaboration. 



\newpage
\section{Task Force 2}
\label{chapter3}

\subsection{Periodic coordination sessions}
\textbf{Objective:} Strengthen communication strategies between
teachers and coordinators from each country 
\\
\textbf{Leading ROAD Institution:} Parque EXPLORA (Colombia), IVIC
(Venezuela). 
\\
\textbf{Lead:} Luz Angela Cubides, Angela Patricia Pérez, Enrique Torres.
\\  
\textbf{Implementation:} Hold bi-weekly coordination sessions. They
have been held during 2014. 

\subsection{Results}
The sessions were active the first months of the Andean ROAD. 
They did not have continuity. 
The biggest challenge was ...

%#######################################################




\subsection{Virtual training sessions - VTS}
\textbf{Objective:} Create virtual training and activities open to
teachers and students. 
\\
\textbf{Leading ROAD Institution:} IVIC (Venezuela).
\\
\textbf{Lead:} Enrique Torres, Juan Carlos Arias.
\\
\textbf{Implementation.} Develop a virtual platform to hold virtual
workshops on information technologies applied to astronomy and space
sciences. 

\subsubsection{Results}

We held some meetings with the TF2 (Children and Schools) and TF3
(Public Disclosure) working groups with representatives of the Andean
ROAD. 
These meetings did not have continuity after 2017.



%#######################################################



\subsection{Teaching material for visually impaired students - TMVS}
\textbf{Objective:} Design teaching material to work with visually impaired students.
\\
\textbf{Leading ROAD Institution:} Planetario de Bogotá (Colombia).
\\
\textbf{Contact:} Angela Patricia Perez, Dilia Gonzalez.
\\
\textbf{Implementation:} We start by doing research on the kind of
material to develop. Later on we will define the production and
distribution processes. 


\subsubsection{Results}

%#######################################################


\subsection{Annual TF2 meeting}
\textbf{Objective:} Organize and annual meeting to gather all the collaborators in the Task Force.
\\
\textbf{Leading ROAD Institution:} Changes every year.
\\
\textbf{Lead:} Maritza Arias Manriquez, Jonathan Moncada, Manuel de la Torre.
\\
\textbf{Implementation:} Every year we will hold a meeting aimed at collaborators in the node. The venue will change every year. The first meeting will take place in 2015 with the theme Archeoastronomy.

\subsubsection{Results}

The project did not make any significant progress towards reaching its
objective.

%---------The scope of this chapter

\newpage
\section{Task Force 3}
\label{chapter 4}


\subsection{Development for planetariums - DFP}
\textbf{Objective:} Develop special shows for planetariums in the region.
\\
\textbf{Leading ROAD Institution:} Parque EXPLORA (Colombia).
\\
\textbf{Contact:} Carlos Molina.
\\
\textbf{Implementation:} We want to develop special shows to be projected at planetariums and science museums from the region, highlighting ancestral traditions from indigenous tribes and founding communities from South America.
These shows would be shared between the Local and Regional Networks of Planetariums, such as “Asociación de Planetarios
del Cono Sur”, and the growing “Red de Planetarios de Colombia”.

\subsubsection{Achievements}

We only succeded to perform initial contacts.
We did not manage to develop the special shows.


Representatives of Parque Explora (Medellín, Colombia) got in touch with chilean colleages to develop ideas to create shows for planetariums in the Andean region.
There was a plan to implement a pilot plan in 2017-2018 for a digital planetarium to be located in San Pedro de Atacama.
Unfortunately this did not came to fruition.
The National Polytechnic School in Quito had a initial approach with Planetarium in Quito.

\subsection{Communicating astronomy with the public}
\textbf{Objective:} Organize periodic meetings to gather all the collaborators in the Task Force, "Communicating Astronomy with the Public”.
\\
\textbf{Leading ROAD Institution:} SOCHIAS ROAD office (Chile, Oficina Nacional de Coordinación, ONC).
\\
\textbf{Contact:} Farid Char.
\textbf{Implementation:} We want to hold regular meetings that can serve as an exchange platform for people involved in communicating astronomy to the public; this includes science museums, planetariums, mass media and associated industries. We also want to use this opportunity to showcase efforts in the other two task forces. This meeting will be
held parallel to large professional meetings. We envision this event as a ”Communicating Astronomy with the Public” meeting for the region. We expect to hold the first one during the next Latin-American Regional IAU Meeting in 2016.


\subsubsection{Results}
We implemented a one-afternoon mini-worskshop during the Latin-American Regional IAU Meeting in 2016.
Parque Explore hosted the Communicating Astronomy with the Public meeting in 2016. Colleagues from the Andean ROAD attended,
but we did not manage to stablish a similar regular meeting for the Andean Region.

\newpage
\section{Other Achievements}

\textbf{Farid Char,  Coordinator of Outreach Centro de Astronomía, U. de Antofagasta.}

\begin{itemize}
    \item Participation in 1st Face-to-Face Meeting with representatives of OAD Regional Offices in Cape Town, South Africa (January 2015).

    \item Participation in quarterly virtual meetings with representatives of OAD Regional Offices for updates, news and future actions.

    \item Preparation of 2 educational proposals for Call for Proposals OAD 2015 (1 standby proposal, “Wish list”)

    \item Participation in the 1st Chile-US Summit on Astronomy Education (oral presentation and poster on the Node), in Santiago and San Pedro de Atacama, March 2015.

    \item Establishment of work networks and contact with representatives of American universities to explore collaborations with Latin American communities (U. Arizona, Green Bank Observatory).

    \item Participation in the 2nd Astronomy Workshop in the Andes (talk about astronomy in Chile and postulation of the country as the headquarters of the IAU General Assembly in 2021), and Chilean representation in the signing of the agreement to create the Andean Node in Bogotá, Colombia , July 2015.

    \item Strategic contact with the Regional Unit of International Affairs (UREA) of the GORE Antofagasta, to develop activities and presentations that allow to know the Andean Node in countries that integrate the Node (in principle, ZICOSUR countries).

    \item Participation in the Science and Technology Commission of the Integration Committee between the Northeast of Argentina and the North of Chile, to publicize the Node and explore alternative activities that integrate ZICOSUR countries, in Iquique, August 2015.

    \item Participation in the 58th Annual Meeting of the Argentine Astronomy Association, where they participated in conversations with representatives of the AAA (equivalent to SOCHIAS of Argentina), to integrate as an active country to the Andean Node, in La Plata, September 2015.

    \item Participation in the IAU General Assembly and participation in the OAD working group, through the award of an IAU scholarship (poster presentation and oral presentation), in Honolulu, USA, August 2015.

    \item Integration to 3 Working Groups of IAU Commissions to strengthen participation of institutions and visibility of Latin American initiatives (C2 WG CAP Journal, C1 WG Network for Astronomy School Education (NASE) and C2 WG Outreach Professionalization \& Accreditation), August 2015.

    \item Strategic contact with initiative UAbierta of the U. of Chile, to explore requirements in giving a Massive Online Open Course (MOOC) on general astronomy open to participants from South America, with emphasis on Node countries, January 2016.

    \item Online contributions to generate the OAD Business Plan for 2016-2021, together with representatives of other OAD Regional Offices, February 2016.

    \item Participation in the 2nd Chile-US Summit on Astronomy Education (integration to the Roadmap document editor team) in La Serena, March 2016.

    \item Sending a proposal on behalf of the Andean Node for financing a NASE school in Chile to the Gemini Fund 2016 (not selected). - Strategic contacts with Latin American institutions and teachers, to probe expressions of interest in using NOAO Quality Lightning Teaching Kits obtained by the Andean Node, to measure the conditions of light pollution in different countries of the Node, April 2016.

    \item Strategic contacts with teachers and Latin American institutions, in order to coordinate joint activities for the observation of the transit of Mercury, May 2016.

    \item Participation in the Communicating Astronomy with the Public event (CAP 2016) in Medellín, Colombia (May 2016).

    \item Participation in a working table with representatives of National Olympics in different countries of the Node, to explore strategies together to improve the Latin American Astronomy Olympics and optimize the level of the participants in each edition, as well as take ideas applicable to the Olympics in Chile , during the CAP2016 event.

    \item Preparation of 2 educational proposals for Call for Proposals OAD 2016 (not selected), July 2016.

    \item Contact with a group of Uruguayan teachers who ask for guidance and ideas on educational activities to be carried out with Chilean peers during a trip to Chile as a training instance; They are referred to the network of teachers in Chile and are followed up, September 2016.

    \item Participation as part of the evaluation committee (pre-selection) for proposals sent to the OAD Call for Proposals, September 2016.

    \item Strategic collaboration with a letter of commitment to contribute to the project "Uniting Chile, Latin America and the world through Astro-Engineering" for an Astro-Engineering School in the University of Chile, September 2016.

    \item Establishment of work networks to develop educational materials with representatives of UNAWE and Planetarium of Medellín, since May 2016 (in preparation).

    \item Development of an intervention plan for the Astronomy Olympics in Chile, so that SOCHIAS takes full administration of this event, October 2016.

    \item Administration of the Astronomy Olympics in Chile 2017, through the coordination of its 2 stages: Regional Olympiad, virtual (already made in March 2017), and National Olympiad, to be held in person in September 2017.

    \item Preparation of 3 educational proposals for Call for Proposals OAD 2017 (sent March 2017, awaiting response).

    \item Pre-registration to participate as an exhibitor in the Communicating Astronomy with the Public event (CAP 2018) and travel grant application. Fukuoka, Japan (next, March 2018).

    \item Pre-registration to participate as an exhibitor at the 3rd Chile-USA Summit. on Education in Astronomy (oral presentation), in Santiago (next, August 2017).

    \item Application to participate as an exhibitor (oral presentation) in the ITCA NARIT Colloquium, “Astronomy for STEM Education” that selects 100 astronomy educators worldwide, and includes OAD participation in its program of activities. Bangkok, Thailand (next, August 2017).

    \item Participation as part of the evaluation committee (pre-selection) for proposals sent to the OAD Call for Proposals, July 2017.
\end{itemize}    
\newpage

\textbf{ Giovanny Cardona UDistrital}

From the Distrital University in Bogotá Colombia se ha apoyado estos eventos realizado publicaciones y concluido tesis en Enseñanza de la Astronomía se dejan a consideración si pueden ser parte del informe
IX Congreso Nacional de Enseñanza de la Física y la Astronomía
El 9 CONGRESO NACIONAL DE ENSEÑANZA DE LA FÍSICA Y LA ASTRONOMÍA es un espacio de encuentro para reflexionar sobre la Investigación en enseñanza de la Física y la Astronomía en Colombia, considerando las condiciones de diversidad cultural propias de los contextos en donde se desarrolla en diálogo con las perspectivas internacionales en esta materia. En el marco de este 9 Congreso, tendrá lugar el simposio: “Historiadoras de las ciencias aportes a la formación de Licenciados en Física” escenario para la construcción de alternativas y fortalecimiento de los procesos de renovación de los programas de Licenciatura en Física.

Este evento se celebra cada dos años y reúne docentes de física de todos los niveles educativos, investigadores en enseñanza de la física, entes gubernamentales y empresariales relacionados directamente con la educación; todos interesados en compartir, reflexionar y discutir en torno a propuestas, practicas, nuevos enfoques, estrategias y teorías de la problemática educativa de la enseñanza y el aprendizaje de la física, que se adelantan en los diferentes contextos educativos del país.
En el 9 CNEF se manifiesta un variado espectro de opciones, reflexiones, propuestas, iniciativas, e investigaciones y cuenta con la participación de destacados invitados de orden nacional e internacional, que se constituyen en uno de los centros de interés de la dinámica académica que se vive durante su realización. Los logros, dificultades, obstáculos y elaboraciones de la comunidad permiten evidenciar líneas de investigación que definen y caracterizan la comunidad de educadores en física, a partir de fundamentos históricos y epistemológicos de la enseñanza, el aprendizaje y la didáctica de la física, así como también entrar a revisar los fundamentos de prácticas, procesos y metodologías de investigación.

De esta forma, cada participante encuentra opciones para cambiar, contextualizar y transformar la forma de ver el mundo del aula desde la física y los fenómenos que se presentan en ella; esencia fundamental de la comunidad de profesionales en docencia de la física.
En el marco de esta versión del Congreso, tendrá lugar el simposio: “Historiadoras de las ciencias aportes a la formación de Licenciados en Física”, como escenario para la construcción de alternativas y fortalecimiento de los procesos de renovación de los programas de Licenciatura en Física.
FECHA:Noviembre 7, 8 y 9 de 2018
Lugar: Universidad Antonio Nariño. Sede Nicolás de Federmán. Calle 58 A \#37-94 Bogotá, Colombia
CONVOCAN: Asociación Colombiana de Profesores de Física-ZEMAKAITA, Universidad Distrital Francisco José de Caldas, Universidad Pedagógica Nacional
APOYAN: Universidad Antonio Nariño, Universidad de Nariño, Universidad del Valle, Universidad Nacional de Salta-Argentina, Red de profesores del Distrito Capital. Nodo de Ciencias y Matemáticas, Secretaria de Educación de Cundinamarca, Secretaria de Educación de Bogotá.

\newpage
\section{Venezuela}

In Venezuela, some fans still subsist, but there are only 4 astronomers in exercise, 3 women, 1 man, 1 of the women is retired and the other 3 without resources to continue working in the country and considering emigrating. Students have 2 masters left at the moment doing theses linked in some way to CIDA (both tutorsed by former CIDA researchers, one in Uruguay and the other in Mexico). Of rest, I am not aware of any other thesis in astronomy / astrophysics of any university that is being carried out in Venezuela.
\newpage
\section{Conclusions}



%bibliography
%\bibliographystyle{abbrv}
%\bibliography{sample.bib}

\end{document}
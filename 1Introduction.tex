\newpage
\section{Introduction}

In the proposal to start the Andean Regional Office of Astronomy for
Development (ROAD), submitted on November 2, 2014, we wrote the following:

\begin{quote}
The countries in the Andean Region (Bolivia, Colombia, Chile,
 Ecuador,Peru and Venezuela) represent a common language block in
 South America. 
 Historically they have also made part of common trading blogs (Andean
 Community, Comunidad Andina (CAN) in Spanish). 
They share similar social conditions and goals for scientific and
 human developmen development. 


The astronomical development in each of these countries can be
efficiently achieved through regional cooperation. 
With the leadership of Chile in aspects of professional astronomy and
Colombia for public outreach, it is possible to develop strategies to
strengthen the professional research, education and popularization of
astronomy in the Andean region. 

This development effort requires the commitment and shared effort from
different institutions. 
An Andean Regional Office of Astronomy for Development (ROAD) can
serve this goal. It will strengthen ongoing collaboration efforts,
create channels of communication and develop new strategies to
exchange knowledge and human resources in the region.  


The participating institutions have been involved in different
projects in each one of the ROAD core areas: research, schools and
children and public. However, the participation and degree of
commitment has not been homogeneous across the region, or even inside
the same country. 


Therefore our main goal is to guarantee and strengthen effective
methods of communication between the representatives and coordinators
of global, regional and local projects implemented in the Andean
countries, specially looking to orient and advise new working groups 
in other cities and regions. This lead us to define the following
objectives

\begin{itemize}
\item Foster the goals of the International Astronomical Union (IAU)
Strate- gic Plan in the Andean region. 

\item Serve as partner to the Office of Astronomy for Development (OAD),
the IAU and other international organizations to plan and implement
relevant projects in the Andean region. 
\item Create public forums where the project management of all
development activities can be communicated and evaluated. 
\item Initiate and coordinate fund-raising activities for regional
development activities. 
\item Create new institutional alliances among countries in the region
to exchange knowledge and human resources. 

\end{itemize}

\end{quote} 

To reach those objectives we suggested a series of specific projects and tasks. 
In this report we evaluate to what extent those task were succesfuly
completed.

The following score card grades each task from 0 (unfulfilled) to 10
(completely fullfilled). We give a grade of $2$ to tasks that were
started but did not continue. We give a grade of $6$ to tasks that
advanced and gave positive results only for a fraction of the
institutions participating in the Andean ROAD.

\section{Score Card}

\begin{table}[!h]
\begin{center}
\begin{tabular}{lp{9cm}r}
Task Force & Project & Score (0-10)\\\hline
1 & Andean School On Astronommy Astrophysics & 6\\
1 & Andean Graduate Program & 0 \\
1 & Exploration Working Group in Astroparticle Physics & 6\\
1 & Exploration Working Group in Radioastronomy & 6\\
1 & Massive Open Online Courses & 0 \\ 
1-2-3 & Communitacion Network & 10\\
2  & Periodic Coordination Sessions & 2\\
2 & Virtual Training Sessions & 2\\
2 & Teaching Material for Visually Impaired Students & 6\\
2 & Annual TF2 Meeting & 2\\
3 & Development for planetariums & 0\\
3 & Regional Communicating Astronomy with the Public & 2\\\hline
 & \textbf{Average} & 3.33\\ 
\end{tabular}
\end{center}

In the following section we present more details about each project.
\end{table}


